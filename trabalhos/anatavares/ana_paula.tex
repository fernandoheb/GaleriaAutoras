

<!DOCTYPE html>

<html>
       
 <head>
           
    <meta charset="UTF-8">
        
       <title> Ana Paula Tavares </title>
    
    </head>
     
   <body>
   <header>
      <center>
      <h1>
       
       </h1>
      </center>
   
    <center>
      <h1>
        <mark>Ana Paula Tavares</mark> 
   <hr/>
      </h1>
  </center>
    <h2>Veja</h2>
    <ul>
      <li> <a href="#s1"> Seção 1</a> </li>
      <li> <a href="#s2">Seção 2 </a> </li>
      <li> <a href="#s3"> Seção 3 </a> </li>
    </ul>
    <h3>Visite</h3>
  <ul> 
    <li> <a href="file:///C:/Users/Iasmin/Documents/IFSP/DWS/1%C2%B0Bimestre/galeria_ana.html" targent="blank" title="Fotos da Ana Paula Tavares"> Galeria Ana Paula Tavares </a> </li>
	</ul>

    <center>  

  <img src="https://th.bing.com/th/id/R1d5519d44c2baa327df236cc84d1dcc3?rik=XwBWEeR2vq41Zw&riu=http%3a%2f%2f2.bp.blogspot.com%2f_HwEfv-Phgm8%2fTNgwXswcMYI%2fAAAAAAAAAw0%2fSqFigiCyDPM%2fs1600%2fana%2bpaula.jpg&ehk=QDbJiLZI%2bIp06fdhActLFLEQuDJ2zh5Qilygg3ua1DY%3d&risl=&pid=ImgRaw" alt="Altora" width="250px" height="auto">
      </center>
 

    <h2 id="s1"> Biografia </h2>
 <p>
  <strong> Ana Paula Ribeiro Tavares  </strong> Nascida no Lubango em Huila, provincia da Huila, Sul de Angola(continente africano), foi criada por padrinhos, tendo ali vivido ate aos 20 anos.
  Estudou Historia na Faculdade de Letras de Luanda e terminou em Lisboa. Após o casamento viveu no Huambo, Cuanza Sul, Benguela e finalmente Luanda. Ana Paula Tavares vem atuando em varias atividades ligadas a literatura e a historia africana. 
     </p>
	  <a href="#"> voltar </a>
	  <hr>
	  

    <h2>   Todas as suas obras: </h2>
  <ul> 

<li> Sangue da buganvilia: crônicas (prosa). <strong> - 1998 </strong> </li> 
<li> O Lago da Lua (poesia). <strong> - 1999 (ano da primeira edicao e publicacao) </strong> </li>
<li> Dizes-me coisas amargas como os frutos (poesia). - <strong> 2001. </strong> </li>
<li> Ex- votos.- <strong> 2003 </strong> </li>
<li> A Cabeça de Salome (prosa). - <strong> 2004 </strong> </li>
<li> Os olhos do homem que chorava no rio (romance).Produzido por Ana Paula Tavares e Manuel Jorge Marmelo. - <strong> 2005 </strong> </li>
<li> Cerimonia di passaggio (poesia). <strong> - 2006 </strong>
<li> Ritos de passagem.<strong> -2007 </strong> (ano publicado) </li>
<li> Manual Para Amantes Desesperados (poesia). - <strong> 2007 </strong> </li>
<li> Contos de vampiros - <strong> 2009 </strong> </li>
<li> Como veias finas na terra (poesia). - <strong> 2010 </strong> </li>
<li> Fieberbaum: Arvore da febre - <strong> 2010 </strong> </li>
<li> Amargos como os frutos: poesia reunida - <strong> 2011 </strong> </li>
<li> Verbetes para Um Dicionario Afetivo - <strong> 2016 </strong>
<li> Manuale per amanti disperati - <strong> 2017 </strong> </li>
<li> La testa di Salome - <strong> 2017 </strong> </li>
<li> Um rio preso nas maos: crônicas - <strong> 2019 </strong> </li>

 </ul>
  
    <h2 id="s2" > Suas principais obras foram: </h2>
	 <h3>Visite</h3> 
  <ul>
 <li> <a href="file:///C:/Users/Iasmin/Documents/IFSP/DWS/1%C2%B0Bimestre/obras_ana.html" target="blank" title="Obras de Ana Paula Tavares"> Obras de Ana Paula Tavares </a> </li>
 </ul>

<ul>
  <li>  Ritos de passagem.<strong> - 2007 </strong> (OBS: em 1985 ocorreu sua primeira edição) </li>
  <li>  Sangue da buganvilia: crônicas (prosa). <strong> - 1998 </strong> </li> 
  <li>  O Lago da Lua (poesia). <strong> - 1999 </strong> </li>
  <li>  Dizes-me coisas amargas como os frutos (poesia). - <strong> 2001. </strong> </li>
  <li>  Ex- votos.- <strong> 2003 </strong> </li>
  <li>  A Cabeça de Salome (prosa). - <strong> 2004 </strong> </li>
  <li>  Os olhos do homem que chorava no rio (romance).Produzido por Ana Paula Tavares e Manuel Jorge Marmelo. - <strong> 2005 </strong> </li>
  <li>  Manual Para Amantes Desesperados (poesia). - <strong> 2007 </strong> </li>
  <li>  Como veias finas na terra (poesia). - <strong> 2010 </strong> </li>
  <a href="#"> voltar </a>
    <hr>  
 </ul>
</a>
 <h2> Suas graduações: </h2>

 <ul> 

<li> Possui Bacharelado e Licenciatura em Historia pela Universidade de São Paulo - <strong>  (1995). </strong> </li>
<li> Mestrado em Historia Social pela Universidade de São Paulo - <strong>  (1998). </strong> </li>
<li> Doutorado em Historia Social pela Universidade de São Paulo - <strong>  (2003). </strong> </li>
  </ul>
  
  <h2> Influências </h2>
  <p>
  A escrita de Ana Paula Tavares sofreu influência de autores brasileiros, como Manuel Bandeira, Jorge Amado, Carlos Drummond de Andrade e João Cabral de Mello Neto, cujas obras chegavam a Angola por meio de viajantes. Segundo a poeta, não só a literatura, mas também a música brasileira influenciou sua escrita.
</p>

<h2 id="s3">Prêmios: </h2>
<h3>Visite</h3>
<ul>
<li> <a href="file:///C:/Users/Iasmin/Documents/IFSP/DWS/1%C2%B0Bimestre/premios_ana.html" targent="blank" title=" Prêmios de Ana Paula Tavares">Prêmios de Ana Paula Tavares </a> </li>
<li> 2007 – Prêmio Nacional de Cultura e Artes, secção de Literatura, Angola, pelo livro Manual para amantes desesperados. Lisboa: Caminho, 2007.
<li> 2004 – Prêmio Mário António da Fundação Calouste Gulbenkian, pelo livro Dizes-me coisas amargas como os frutos. Lisboa: Caminho, 2001
</ul>
<a href="#"> Voltar </a>
 
 <h3> Ana Paula Tavares não tem obras que tenham sido adaptados para filmes,teatro ou pinturas. </h3>
 
 
   

 
           
 <h3>  Fontes:  </h3> 

 <small>
 <ol>
            
 <li> https://www.portalsaofrancisco.com.br/biografias/ana-paula-ribeiro tavares#:~:text=Ana%20Paula%20Ribeiro%20Tavares%20nasceu,3%20de%20outubro%20de%201952..&text=Foi%20membro%20da%20Uni%C3%A3o%20dos,poesia%20intitulado%2C%20RITOS%20DE%20PASSAGEM.
 
  <li> https://pt.wikipedia.org/wiki/Ana_Paula_Tavares#Biografia

 <li> https://www.google.com.br/search?q=Ana+Paula+Tavares+Livros&spell=1&sa=X&ved=2ahUKEwjq7tKCodbwAhUsq5UCHYo-AFkQBSgAegQIARA6&biw=1517&bih=694 </li>
                      
                       
   <h3>   Referencias da Wikipedia :   </h3>
                     
<li> <a href="https://pt.wikipedia.org/wiki/Ana_Paula_Tavares" targent="blank" > Ana Paula Tavares </a> </li>

 <li>  Infopedia. Porto: Porto Editora, 2003-2013
 <li>  Marta Lanca (24 de Junho de 2019). "nao posso escorregar na emocao facil, que a saudade e a distancia criam", entrevista a Ana Paula Tavares. Consultado em 16 de Julho de 2019 </li>
 <li>  Universidade Nova de Lisboa (2010). Repositorio da Universidade Nova de Lisboa. Consultado em 16 de Julho de 2019 </li>
 <li>  O Arquetipo Feminino Em Quatro Poemas Da Serie "Mukai" de Ana Paula Tavares. Erica Antunes Pereira, Uniao dos Escritores Angolanos. </li>
 <li>  Canto de nascimento </li>
 <li>  SOUZA, Mailza R. Toledo. Do corpo ao texto - A mulher inscrita/escrita na poesia de Hilda Hilst e Ana Paula Tavares. Sao Paulo: USP, 2009; p.150 </li>
 <li>  SOUZA, Mailza R. Toledo. Do corpo ao texto - A mulher inscrita/escrita na poesia de Hilda Hilst e Ana Paula Tavares. Sao Paulo: USP, 2009; p.150 </li>
 <li>  Poesia africana. Ana Paula Tavares </li>
 <li>  A abobora menina </li>
 </ol>
</small>

          <footer> Desenvolvido por Yasmin Vitória </footer> 
      </body>


</html>